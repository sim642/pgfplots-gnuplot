\documentclass{article}

\usepackage{xcolor}
\usepackage{multicol}
\usepackage{pgfplots-gnuplot}

% Copied from xcolor documentation source
\makeatletter

\def\testclr#1#{\@testclr{#1}}
\def\@testclr#1#2{{\fboxsep\z@\fbox{\colorbox#1{#2}{\phantom{XX}}}}}
\def\Testclr#1#{\@Testclr{#1}}
\def\@Testclr#1#2#3{\testclr#1{#2}~\rlap{\Color[-]{#3}}\\}
\def\TestClr#1#{\@TestClr{#1}}
\def\@TestClr#1#2#3{\testclr#1{#2}~\rlap{\Color[+]{#3}}\\}

\makeatother
\newcommand*\Color[2][+]{\textsl{#2}\ifx#1+\index{color names\levelchar\textsl{#2}\string|usage}\fi}


\begin{document}
\begin{multicols}{3}[\section*{Colors}]
\footnotesize\def\0#1{\Testclr{#1}{#1}}
\0{gnuplot-white}
\0{gnuplot-black}
\0{gnuplot-dark-grey}
\0{gnuplot-red}
\0{gnuplot-web-green}
\0{gnuplot-web-blue}
\0{gnuplot-dark-magenta}
\0{gnuplot-dark-cyan}
\0{gnuplot-dark-orange}
\0{gnuplot-dark-yellow}
\0{gnuplot-royalblue}
\0{gnuplot-goldenrod}
\0{gnuplot-dark-spring-green}
\0{gnuplot-purple}
\0{gnuplot-steelblue}
\0{gnuplot-dark-red}
\0{gnuplot-dark-chartreuse}
\0{gnuplot-orchid}
\0{gnuplot-aquamarine}
\0{gnuplot-brown}
\0{gnuplot-yellow}
\0{gnuplot-turquoise}
\0{gnuplot-grey0}
\0{gnuplot-grey10}
\0{gnuplot-grey20}
\0{gnuplot-grey30}
\0{gnuplot-grey40}
\0{gnuplot-grey50}
\0{gnuplot-grey60}
\0{gnuplot-grey70}
\0{gnuplot-grey}
\0{gnuplot-grey80}
\0{gnuplot-grey90}
\0{gnuplot-grey100}
\0{gnuplot-light-red}
\0{gnuplot-light-green}
\0{gnuplot-light-blue}
\0{gnuplot-light-magenta}
\0{gnuplot-light-cyan}
\0{gnuplot-light-goldenrod}
\0{gnuplot-light-pink}
\0{gnuplot-light-turquoise}
\0{gnuplot-gold}
\0{gnuplot-green}
\0{gnuplot-dark-green}
\0{gnuplot-spring-green}
\0{gnuplot-forest-green}
\0{gnuplot-sea-green}
\0{gnuplot-blue}
\0{gnuplot-dark-blue}
\0{gnuplot-midnight-blue}
\0{gnuplot-navy}
\0{gnuplot-medium-blue}
\0{gnuplot-skyblue}
\0{gnuplot-cyan}
\0{gnuplot-magenta}
\0{gnuplot-dark-turquoise}
\0{gnuplot-dark-pink}
\0{gnuplot-coral}
\0{gnuplot-light-coral}
\0{gnuplot-orange-red}
\0{gnuplot-salmon}
\0{gnuplot-dark-salmon}
\0{gnuplot-khaki}
\0{gnuplot-dark-khaki}
\0{gnuplot-dark-goldenrod}
\0{gnuplot-beige}
\0{gnuplot-olive}
\0{gnuplot-orange}
\0{gnuplot-violet}
\0{gnuplot-dark-violet}
\0{gnuplot-plum}
\0{gnuplot-dark-plum}
\0{gnuplot-dark-olivegreen}
\0{gnuplot-orangered4}
\0{gnuplot-brown4}
\0{gnuplot-sienna4}
\0{gnuplot-orchid4}
\0{gnuplot-mediumpurple3}
\0{gnuplot-slateblue}
\0{gnuplot-yellow4}
\0{gnuplot-sienna1}
\0{gnuplot-tan1}
\0{gnuplot-sandybrown}
\0{gnuplot-light-salmon}
\0{gnuplot-pink}
\0{gnuplot-khaki1}
\0{gnuplot-lemonchiffon}
\0{gnuplot-bisque}
\0{gnuplot-honeydew}
\0{gnuplot-slategrey}
\0{gnuplot-seagreen}
\0{gnuplot-antiquewhite}
\0{gnuplot-chartreuse}
\0{gnuplot-greenyellow}
\0{gnuplot-gray}
\0{gnuplot-light-gray}
\0{gnuplot-light-grey}
\0{gnuplot-dark-gray}
\0{gnuplot-slategray}
\0{gnuplot-gray0}
\0{gnuplot-gray10}
\0{gnuplot-gray20}
\0{gnuplot-gray30}
\0{gnuplot-gray40}
\0{gnuplot-gray50}
\0{gnuplot-gray60}
\0{gnuplot-gray70}
\0{gnuplot-gray80}
\0{gnuplot-gray90}
\0{gnuplot-gray100}
\end{multicols}

\section*{Markers}

% Copied from pgfplots documentation source

\newenvironment{longdescription}[0]{%
    \begin{list}{}{%
        % \leftmargin=4.7cm
        % \setlength{\labelwidth}{4.7cm}%
        \leftmargin=3cm
        \setlength{\labelwidth}{3cm}%
        \renewcommand{\makelabel}[1]{\hfill\textbf{\texttt{##1}}}%
    }%
}{%
    \end{list}%
}%
\def\showit#1{%
    \tikz\draw [
        gray,
        thin,
        mark options={fill=yellow!80!black,draw=black,scale=2},
        x=0.8cm,y=0.3cm,
        #1,
    ] plot coordinates {(0,0) (1,1) (2,0) (3,1)};%
}%
\def\showitpgfplots#1{%
\begin{tikzpicture}[baseline]
    \begin{axis}[
        anchor=north,
        xticklabels=,
        yticklabels=,
        zticklabels=,
        width=5cm,
    ]
        \addplot3 [
            gray,
            thin,
            mark options={
                scale=2,
                fill=yellow!80!black,
                draw=black,
            },
            #1,
        ] coordinates {(0,0,0) (0.3,0.6,0.3) (2,0,0.1) (2.3,1,0.2)};
    \end{axis}
\end{tikzpicture}%
}%
\newcommand*{\showitem}[1]{\item[mark=#1] \showit{mark=#1}} % Simmo
\begin{multicols}{2}
\begin{longdescription}
    \showitem{gnuplot-1}
    \showitem{gnuplot-2}
    \showitem{gnuplot-2*}
    \showitem{gnuplot-3}
    \showitem{gnuplot-3*}
    \showitem{gnuplot-4}
    \showitem{gnuplot-5}
    \showitem{gnuplot-6}
    \showitem{gnuplot-7}
    \showitem{gnuplot-8}
    \showitem{gnuplot-9}
    \showitem{gnuplot-10}
    \showitem{gnuplot-11}
    \showitem{gnuplot-12}
    \showitem{gnuplot-13}
    \showitem{gnuplot-14}
    \showitem{gnuplot-15}
\end{longdescription}
\end{multicols}

\end{document}
